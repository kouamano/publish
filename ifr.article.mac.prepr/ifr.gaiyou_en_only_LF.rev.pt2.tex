%\begin{document}

%\textgt{Purpose:}
%We analyze the mutual impact relationships among various research fields in the Natural Sciences using a field classification with a large granularity.
%Here, Natural Sciences include application fields such as engineering, medicine, and agriculture.
%On the basis of the classification used in this study, inter-field citations account for approximately 30\% of the total citations.
%The objective of this study is to reveal the research structure of the Natural Science field by analyzing in detail the inter-field citations.
%\textgt{Data Source:}
%The data source of this study is Journal Citation Reports (JCR) 2004 Science Edition (CD-ROM).
%From this data source, we selected 5964 journals that contained articles published in 2000--2004 and cited at least once in 2004. Using the selected journals, we constructed an inter-journal citation matrix, from which we then constructed an inter-field citation matrix.
%\textgt{Method:}
%We used the field classification consisting of 21 fields; this classification is the same as that developed in our previous study, except that this classification united an extremely small field to the nearest field.
%The inter-field citation matrix based on these 21 fields was used for the analyses of the following parameters:
%1. The extent that citations in the individual fields were made from/toward other fields.
%2. Hierarchical relation of the 21 fields.
%3. Impact strength between each field pair (symmetric/asymmetric linkage).
%4. Total influence by the individual fields on other fields (based on the field Eigenfactor indicator).
%\textgt{Results:}
%In general, biomedical fields had a larger size than others, in terms of the number of articles.
%A synergistic effect was recognized between the size of a field and the impact of the articles in the field, by considering the Spearman rank correlation between them.
%There was a weaker synergistic effect between the size of a field and the cited/citing ratio of the field with other fields.
%The result of hierarchical clustering among the fields based on their citation patterns revealed interesting information about the inter-field relationships; for example, analytical chemistry was found to be nearer to agriculture than to other chemical fields.
%The result of symmetric linkage mostly supported the hierarchical structure; however, we observed some differences; for instance, the similarity between the condensed matter physics field and the chemistry field was the highest in the hierarchical structure but not so high in the symmetric linkage.
In the analysis of the field influence based on the Eigenfactor, a large influence of biomedical fields was observed. Additionally, we analyzed the effect of $\alpha$, which tunes the influence ratio between publications and citations, and observed that $\alpha$ is an important parameter for determining the influence ranking of the fields.

%\end{document}
